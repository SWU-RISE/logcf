\section{A new algorithm of computing upper bounds}
\label{sec:alg}
\rev{{\em Theorem \ref{thm:log}} is good theoretical result, it can easily guarantee a value is one of the real roots upper bound for giving polynomial. But it is difficult to directly use it to compute the optimal upper bound. So,  in this section we will give a new algorithm of computing upper bounds. The correction of this new algorithm is guarantee by the theoretically
	results in section \ref{sec:thm}. And this algorithm  has good computability, more importantly is that the result of the algorithm is at most two times greater than the optimal result of {\em Theorem \ref{thm:log}}}.

\rev{In symbolic computing procedure, $1$ is a special value,   {\em Algorithm \ref{alg:less}} is based on {\em Theorem \ref{thm:log}} to check whether $1$ is 
one of upper bound for a given polynomial. } 


\begin{algorithm}
	\SetAlgoCaptionSeparator{.}
	\caption{\less \label{alg:less}}
	\DontPrintSemicolon
	\KwIn{ $p=a_nx^n+a_{n-1}x^{n-1}+\cdots+a_1x+a_0\in\ZZ[x], a_n>0,\ \exists a_i,a_na_i<0 $. }
	\KwOut{ true: the positive root bound of $p$ must be less than $1$;\\\ \ \ \ \ \ \ \ \ \ \ \ \ \    false: cannot determine whether the bound is less than $1$. }
	
	$start=n-1$;
	$lastNeg=0$;\;
	
	\While{$a_{lastNeg}\ge 0$ }{
		$lastNeg=lastNeg+1$;\;
	}
	
	\While{$  a_{start}\ge  0 $ }{
		$start=start-1$;\;
	}
	$cfSum=a_n$;
	$i=n-1$;
	$j=start$;\;
	\While{$i\ge lastNeg-1 \text{ and } j\ge lastNeg-1$ \label{one:start}}
	{
		\If{$cfSum<0$}
		{
			\While{$i>j\text{ and } a_i\le 0$ }{
				$i=i-1$;
			}
			
			\lIf{$i= = j$}{
				\Return {false;\label{one:return}}
			}
			$cfSum=cfSum+a_i$;
			$i=i-1$;\;
			
		}\Else{
		\lIf{$j= = lastNeg-1$ }{
			\Return{ true;}
		}
		\While{$j\ge lastNeg \text{ and }a_j\ge 0 $ }{
			$j=j-1$;
		}
		$cfSum=cfSum+a_j$; $j=j-1$;\;
		\label{one:end}
	}
}
\Return {true; }

\end{algorithm}
The reasons of {\em Algorithm \ref{alg:less}}'s right as follow: 
W.l.o.g. set $a_n>0$ then after program reaches {\em Algorithm \ref{alg:less}}'s line $6$ we have \[\underbrace{a_n}_{+},\underbrace{\underbrace{a_{n-1}}_{+_0}}_{i},\underbrace{\cdots}_{+_0,\cdots,+_0},\underbrace{a_{start+1}}_{+},\underbrace{\underbrace{a_{start}}_{-}}_{j},\underbrace{a_{start-1}}_{*},\underbrace{\cdots}_{*,\cdots,*},\underbrace{a_{lastNeg+1}}_{*},\underbrace{a_{lastNeg}}_{-},\underbrace{a_{lastNeg-1}}_{+_0},\underbrace{\cdots}_{+_0,\cdots,+_0},\]
where "$+$" denotes the corresponding $a_k$ is positive, "$-$" denotes the corresponding $a_k$ is negative, "$*$" denotes the sign of the
corresponding $a_k$ is unknown and  "$+_0$" denotes the corresponding $a_k$ is nonnative. 
The loop between line \ref{one:start} and \ref{one:end}  in {\em Algorithm \ref{alg:less}} program  \rev{shifts $i$ and $j$ right} until one of them reaches value $lastNeg-1$. If $j$ firstly  reaches value $lastNeg-1$ at line $14$,  it is easy to check assert $i\ge j\wedge cfSum\ge 0$ always holds in this iteration. In other words, $\sum_{k=l}^na_k\ge 0$ for \rev{$l=lastNeg,\cdots,n$}. As $a_{k}>0$ for $k=0,\cdots, lastNeg-1$, $\sum_{k=l}^na_k\ge 0$ for \rev{$l=0,1,\cdots, n$}. Apply {\em Theorem  \ref{thm:log}},  $1$ is one of upper bound  of $p$'s positive real roots.


\rev{In {\em Example \ref{ex:exp1}}, in beginning, $cfSum=1$ and the values of  $i$, $j$, $lastNeg$ as follows:
	\[1,\underbrace{2}_{i=5},\underbrace{-4}_{j=4},1,10,\underbrace{-5}_{lastNeg=1},5.\]
	After firstly executing main loop of {\em Algorithm \ref{alg:less}} between line \ref{one:start} and line \ref{one:end}, the values will become as 
	$cfsum=-3$	\[1,\underbrace{2}_{i=5},-4,\underbrace{1}_{j=3},10,\underbrace{-5}_{lastNeg=1},5.\]
		After secondly executing main loop of {\em Algorithm \ref{alg:less}} between line \ref{one:start} and line \ref{one:end}, the values will become as 
		$cfsum=-1$	\[1,2,\underbrace{-4}_{i=4},\underbrace{1}_{j=3},10,\underbrace{-5}_{lastNeg=1},5.\] Hence, during thirdly executing main loop of {\em Algorithm \ref{alg:less}} between line \ref{one:start} and line \ref{one:end},
		then algorithm will return on line \ref{one:return}. 
}

\rev{
 It is difficult to compute the optimal value which satisfies {\em Theorem \ref{thm:log}}. Hence, we provide an {\em Algorithm \ref{alg:up}}, which based on idea of {\em dichotomic search} to find a
	value which is   enough close to the optimal value. The correctness of {\em Algorithm \ref{alg:up}}  is direct based on {\em Theorem \ref{thm:log}}.   {\em Theorem \ref{thm:log}} needs
	a value $u$ satisfies that  $\min_{j=0}^{n}\left\{  \sum_{i=j}^n a_i u^{i-j}\right\}\ge0$. The resulting  $\lambda$ of  {\em Algorithm \ref{alg:up}}   satisfies  that 
	$  \sum_{i=j}^n a_i \lambda^{i-j}\ge 0, j=0,\cdots,n$. So employing {\em Algorithm \ref{alg:up}}  $\lambda$ is one of upper bound for giving polynomial.  Employing {\em Theorem \ref{thm:two}},
	$\lambda$ is at most two times greater than optimal upper bound which satisfies {\em Theorem \ref{thm:log}}. }

\begin{algorithm}[H]
	\SetAlgoCaptionSeparator{.}
	\caption{\up \label{alg:up}}
	\DontPrintSemicolon
	\KwIn{$p=a_nx^n+a_{n-1}x^{n-1}+\cdots+a_1x+a_0\in\ZZ[x], a_n>0,\ \exists a_i,a_na_i<0. $ }
	\KwOut{ an upper bound of the positive roots of $p$. }
	$start=n-1$;
	$lastNeg=0$;
	$base=1$;\;
	
	\lIf{$\neg$\less($p$) }{
		return $2$; \tcc{$2$ is a special \rev{value} which leads {\em Algorithm \ref{alg:cf}} to run line \ref{log:branch} branch.}
	}
	\While{$a_{lastNeg}\ge  0$ }{
		$lastNeg=lastNeg+1$;\;
		
	}
	
	\While{$a_{start}\ge  0$ }{
		$start=start-1$;\;
		
	}
	
	$i=n$;\;
	\While{$i= = n$}{
		
		$i=n-1$;
		$j=start$;
		$cfSum=a_n$;\;
		
		\While{$i\ge lastNeg-1 \text{ and } j\ge lastNeg-1$}{
			\If{$cfSum<0$}
			{
				\While{$i>j\text{ and } a_i\le 0$ }{
					$i=i-1$;
				}
				\lIf {$i= = j$}{
					break;
				}
				$cfSum=cfSum+a_i2^{(n-i)base}$;
				$i=i-1$;\;
			}\Else{
			
			\lIf{$j= = lastNeg-1$ }{
				$j=lastNeg-2$;\tcc{ $lastNeg-2$ is a special value which leads program to run line \ref{return} branch.}
				break;
			}
			
			\While{$j\ge lastNeg \text{ and } a_j\ge 0$ }{
				$j=j-1$;
			}
			
			$cfSum=cfSum+a_j2^{(n-j)base}$;
			$j=j-1$;\;
		}
		
	}
	\lIf{$j= = lastNeg-2$\label{return}}{
		$base=base+1$;
		$i=n$;\;
	}
	
}
\Return{$\frac{1}{2^{base-1}}$; }

\end{algorithm}



\begin{theorem}
	\label{thm:two}
	Let $p(x)=a_nx^n+a_{n-1}x^{n-1}+\cdots+a_0\ (V(p)> 0)$ be a polynomial with real coefficients. Let  $u$ denote the output of {\em Algorithm \ref{alg:up}} and $u_1$ denote the optimal upper bound of $p$ satisfying
	{\em Theorem \ref{thm:log}}. When $u$ is less than or equal to $1$, $u<2u_1$.
	
\end{theorem}


\begin{proof}
	In {\em Algorithm \ref{alg:up}}, if  $\frac{1}{2^{base}}\ge u_1,$ then $\min_{j=0}^{n}\left\{ \sum_{i=j}^na_i\left( {\frac{1}{2^{base}} }
	\right)^{i-j}\right \}\ge 0$ by the proof of {\em Theorem \ref{thm:log}} and thus 
	the loop does not terminate at this step.
	So when {\em Algorithm \ref{alg:up}} returns, $base$ must satisfy $\frac{1}{2^{base}}<u_1$. Therefore, the output $u=\frac{1}{2^{base-1}}$ and $u<2u_1$.
	Obviously, this algorithm will terminate.
	
	Furthermore,  $\min_{j=0}^{n-1}\left\{ \sum_{i=j}^na_i\left( {\frac{1}{2^{base-1}} } \right)^{i-j}\right \}\ge 0$ by {\em Theorem \ref{thm:log}}. So, $u=\frac{1}{2^{base-1}}$ is an upper bound of $p$.
\end{proof}


\begin{corollary}
	Let $p(x)=a_nx^n+a_{n-1}x^{n-1}+\cdots+a_0$ $ (V(p)> 0)$ be a polynomial with real coefficients. Set $u$ to be the optimal upper bound of positive roots of \rev{$p$} satisfying {\em Theorem
	\ref{thm:log}}. Then  {\em Algorithm \ref{alg:up}} costs at most $O(n\log(u+1))$  additions and multiplications.
\end{corollary}






\subsection{Tricks}
{\bf Variable substitution}
If $p(x)\in \ZZ[x]$ and $p(x)=p_1(x^k)\ (k>1),$ then substitute $y=x^k$ in $p$. Obviously, $\deg(p_1,y)=\frac{\deg(p,x)}{k}$. We first isolate the real roots of $p_1$ then
obtain the real roots of $p$. Using this trick, we can greatly reduce the running time of ${\it ChebyshevT}$
and {\it ChebyshevU} when each term of the polynomials is of even degree. The same trick was also taken into account in \cite{johnson06}.

{\bf Incomplete termination check}
If $p(x)\in \ZZ[x]$ and $V(p)= 2$, we may try to check whether the sign of $p(1)$ is the same as the sign of the leading coefficient of $p$. If they are not the same, then $p$  has one positive root in $(0,1)$ and the other one in $(1,+\infty)$. So, we can terminate this subtree. Since the whole \froot\ procedure is a tree and \froot\ spends more than 90 percent of the
total time on computing $T(p)$, this trick may improve the efficiency of the algorithm greatly.