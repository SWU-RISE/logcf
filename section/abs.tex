\begin{abstract}
  %This paper revisits an   algorithm for isolating real roots of  univariate
  %polynomials based on continued fractions. It follows  the work of Vincent,	Uspensky, Collins and Akritas, Johnson and Krandick.
Computing upper bounds of the positive real roots of some polynomials is a key step of those real root isolation algorithms based on continued fraction expansion and Vincent's theorem. We give a new algorithm  for computing an upper bound of positive roots in this paper. The complexity of the algorithm for computing an upper bound of positive roots is  $O(n\log(u+1))$ where $u$ is the optimal upper bound satisfying Theorem \ref{thm:log} of this paper and $n$ is the degree of the polynomial. Our method together with some tricks have been implemented as a  software package \froot\ using \texttt{C} language. For many benchmarks \froot \  is about four  times faster   than  the function \REALROOT\ of \MAPLE, {\color{red}	is as performance as  \inte\ of \MM\ }   and is also much faster than state of art open source real root solvers in many test cases. %Also, we find that there is a bug in  {\tt \inte} during we test for it's inaccurate zero judgment.}
%   another continued fractions based on software \cf, which seems to be one of the
%  fastest available open software for exact real root isolation. For those  benchmarks which  have only real roots, \froot\ is
%  much faster than \sle\ and \eign\ which are based on numerical computation.
~\\
{\bf Keywords: }
univariate polynomial,  real root isolation, continued fractions,  computer algebra
\end{abstract} 