
%\newpage
\section{A new algorithm of computing upper bounds}

One key ingredient of continued fraction based methods is the computation of upper bounds of the positive real roots of some polynomials. We give in Theorem \ref{thm:log} a new characteristic of such upper bounds of univariate polynomials. A new algorithm based on this theorem, Algorithm \ref{alg:up}, is proposed for computing upper bounds of positive real roots.

\begin{algorithm}
\SetAlgoCaptionSeparator{.}
\caption{\algcf}
\DontPrintSemicolon
\KwIn{ A squarefree polynomial $f \in \ZZ[x] \setminus \{0\}$. }
\KwOut{ $roots$, a list of isolating intervals of positive roots of $f$. }

 $roots=\emptyset$;
  $s=V(f)$;\;
%  \If {$s \equiv 0$}{ \Return { $roots$}; }

 % \If {$s\equiv 1$} {\Return { $(0,\infty)$}; }
$intstack=\emptyset$;
$intstack$.add($\{1,0,0,1,f,s\}$);\;

\While{$intstack\neq \emptyset$} {
  $\{a,b,c,d,p,s\}=intstack.$pop();\tcc{pop  the first element}
  $\alpha=\lb(p)$;\;
  \If{$\alpha\ge1$ }{
	$\{ a,c,p \}=\{ \alpha  a,\alpha c ,H_\alpha(p) \}$;
	$\{ b,d,p \}=\{   a+b,c+d ,T(p) \}$;

	\If { $p(0)= = 0$}{
	  $roots$.add$([\frac{b}{d},\frac{b}{d}])$ ;
	  $p=\frac{p}{x}$;\;
	}
	$s=V(p)$;\;
	\If {$s= = 0$}{
	  continue;\;
	}
	\ElseIf{$s= = 1$}{
			
	  $roots$.add($intvl(a,b,c,d)$);
	  continue;\;
	}

  }
  $ \left\{ p_1,a_1,b_1,c_1,d_1,r \right\}=\left\{ T(p),a,a+b,c,c+d,0 \right\}$

\If{$p_1(0)= =0$}{
  $roots$.add($[\frac{b_1}{d_1},\frac{b_1}{d_1}]$);
  $p_1=\frac{p_1}{x};r=1$;\;
}
$s_1=V(p_1)$;
$\left\{ s_2,a_2,b_2,c_2,d_2 \right\}=\left\{ s-s_1-r,b,a+b,d,c+d \right\}$;
%$ s_2 = s- s_1 - r; a_2 = b; b_2 = a + b; c_2 = d; d_2 = c + d$;\;

\If {$s_2>1$ }{
  $ p_2= (x+1)^{\deg(p)}T(p)$;\;
  \If {$p_2(0)= = 0$}{
	$p_2=\frac{p_2}{x}$;
	$s_2=V(p_2)$;\;
  }
}
\If{$s_1= = 1$ }{

	  $roots$.add($intvl(a_1,b_1,c_1,d_1)$);\;
}
\ElseIf{$s_1>1$ }{

$intstack$.add($\{a_1,b_1,c_1,d_1,p_1,s_1\}$);\;
}

\If{$s_2= = 1$}{

	  $roots$.add($intvl(a_2,b_2,c_2,d_2)$);\;
}
\ElseIf{$s_2>1$ }{

$intstack$.add($\{a_2,b_2,c_2,d_2,p_2,s_2\}$);\;
}
}
\label{alg:cf}
\end{algorithm}

\begin{theorem} \label{thm:log}
  Suppose   $p=a_nx^n+a_{n-1}x^{n-1}+\cdots+a_1x+a_0\ (a_n>0)$  is a univariate polynomial in $x$ with real coefficients.  Then  a nonnegative number $u$ is an upper bound of positive roots of $p$ if $u$   satisfies $\min_{j=0}^{n}\left\{  \sum_{i=j}^n a_i u^{i-j}\right\}\ge0$.
\end{theorem}
\begin{proof}
  If $n= =0$, then $p$ is a nonzero constant and any positive number is its upper bound of positive roots.

  Otherwise, if $b>u$,  we claim that $\sum_{i=j}^na_ib^{i-j}> \sum_{i=j}^na_iu^{i-j}$ for any $j= 0,\ldots,n-1$.

  When $j=n-1$, $\sum_{i=n-1}^na_ib^{i-n+1}-\sum_{i=n-1}^na_iu^{i-n+1}=a_n(b-u)>0.$ The claim holds.

  Assume the claim holds  when $j=k$. When $j=k-1$,  $\sum_{i=k-1}^na_ib^{i-k+1}=\left(\sum_{i=k}^na_ib^{i-k}\right)b+a_{k-1} $. By assumption
  $\sum_{i=k}^na_ib^{i-k}>\sum_{i=k}^na_iu^{i-k}\ge0$. Since $b>u\ge0$, $\left(\sum_{i=k}^na_ib^{i-k}\right)b>\left (\sum_{i=k}^na_iu^{i-k} \right)u  $
  and $\sum_{i=k-1}^na_ib^{i-k+1}> \sum_{i=k-1}^na_iu^{i-k+1}$. So  $\sum_{i=j}^na_ib^{i-j}> \sum_{i=j}^na_iu^{i-j}$ for any $j= 0,\ldots,n-1$.


  By the above claim,   $p(b)=\sum_{i=0}^na_ib^i>0$ when  $b>u$. Because $b$ is arbitrarily chosen, $u$ is an upper bound of the positive roots of $p$.

\end{proof}


The following theorem was given by Akritas et al. in  \cite{akr06,akr08}, which computes positive root upper bounds of univariate polynomials.

\begin{theorem}[Akritas-Strzebo\'{n}ski-Vigklas, \cite{akr06}] \label{thm:qup}

  Let $p(x)=a_nx^n+a_{n-1}x^{n-1}+\cdots+a_0\ (a_n>0)$ be a polynomial with real coefficients and let $d(p)$ and $t(p)$ denote the degree and the number of its terms, respectively.

  Moreover, assume that $p(x)$ can be written as\begin{equation}
	p(x)=q_1(x)-q_2(x)+q_3(x)-q_4(x)+\cdots +q_{2m-1}(x)-q_{2m}(x)+g(x)
	\label{eq:1}
  \end{equation}
  where all the coefficients of polynomials $q_i(x)$ $(i=1,2,\ldots,2m)$ and $g(x)$ are positive. In addition,
  assume that for $i=1,2,\ldots,m$ we have
  \begin{equation*}
	q_{2i-1}(x)=c_{2i-1,1}x^{e_{2i-1,1}}+\cdots+c_{2i-1,t_{2i-1}}x^{e_{2i-1,t_{2i-1}}}
  \end{equation*}
  and
  \begin{equation*}
	q_{2i}(x)=b_{2i,1}x^{e_{2i,1}}+\cdots+b_{2i,t_{2i}}x^{e_{2i,t_{2i}}}
  \end{equation*}
  where $e_{2i-1,1}=d(q_{2i-1})$, $e_{2i,1}=d(q_{2i})$, $t_{2i-1}=t(q_{2i-1}),$ and $t_{2i}=t(q_{2i})$   and the exponent of each term in $q_{2i-1}(x)$ is greater than the exponent of each term in $q_{2i}(x)$. If for all indices $i=1,2,\ldots,m$, we have
  \begin{equation*}
	t(q_{2i-1})\ge t(q_{2i}),
  \end{equation*}
  then an upper bound of the values of the positive roots of $p(x)$ is given by

  \begin{equation}\label{eq:2}
	up= \max_{i=1,2,\ldots,m}\left\{ \max_{j=1,2,\ldots,t_{2i}}\left\{ \left( \frac{b_{2i,j}}{c_{2i-1,j}} \right)^{\frac{1}{e_{2i-1,j}- e_{2i,j} } } \right\}  \right\}
  \end{equation}
  for any permutation of the positive coefficients $c_{2i-1,j}, $ $j=1,2,\ldots,t_{2i-1}$.
  Otherwise, for each of the indices $i$ for which we have
  \begin{equation*}
	t_{2i-1}<t_{2i},
  \end{equation*}
  we {\bf break up} one of the coefficients of $q_{2i-1}(x)$ into $t_{2i}-t_{2i-1} +1$ parts, so that now $t(q_{2i} ) = t(q_{2i-1})$ and apply the same formula (\ref{eq:2}) given above.


\end{theorem}

%\textbf{ Theorem \ref{thm:log} , \ref{thm:qup} all describe the properties of upper bounds of positive real roots. We will claim that all the upper bounds obtain by Theorem
%\ref{thm:qup}  also satisfies Theorem \ref{thm:log} by below theorem. So if there has a method which can work out the  optimal upper bound satisfying Theorem \ref{thm:log}, it will at least better than any method based on Theorem \ref{thm:qup} when only consider the sharp of upper bound.}


We shall show in Theorem \ref{thm:com1} that the bound given by Theorem \ref{thm:log} is better than that given by Theorem \ref{thm:qup}.

\begin{theorem}\label{thm:com}
   Let $p(x)=a_nx^n+a_{n-1}x^{n-1}+\cdots+a_0\ (a_n>0)$ be a polynomial with real coefficients and   $u$ denote an upper bound of positive roots of $p$ obtained by Theorem \ref{thm:qup}, then $\min_{k=0}^{n}\left\{  \sum_{i=k}^n a_i u^{i-k}\right\}\ge0$.

\end{theorem}

\begin{proof}
  %Set $u$ obtains after break up and    permutation of the positive coefficients $c_{2i-1,j}, $ $j=1,2,\ldots,t_{2i-1}$ for  $i=1,2,\ldots,m$.

  %For any $k\in n-1,n-2,\ldots,0$.
  For every $ a_i<0$, by Theorem \ref{thm:qup}, there exist  $c_{i_1}x^{e_{i_1}}$ and $b_{i_2}x^{e_{i_2}},$  respectively, such that
    $e_{i_1}>e_{i_2}$ and $c_{i_1}u^{e_{i_1}}\ge b_{i_2}u^{e_{i_2}}$. By  Theorem \ref{thm:qup} $b_{i_2}x^{e_{i_2} }$ is the term $-a_ix^i$
  and $c_{i_1}x^{e_{i_1} }$ is either a whole or a part (broken up by Theorem \ref{thm:qup}) of a positive term of $p$.

  For every  $a_j>0$, by Theorem \ref{thm:qup}, $\left( \sum_{a_i<0,e_{i_1}=j }c_{i_1} \right)\le a_{j}$. So \begin{dmath*}\sum_{i=k}^na_iu^i\ge \sum_{i=k,a_i<0}^n \left( c_{i_1}u^{e_{i_1}}-
  b_{i_2}u^{e_{i_2}} \right)\ge 0 \end{dmath*} for any $k= 0,1,\ldots,n$. Then $\sum_{i=k}^n a_i u^{i-k}\ge0 $ for any  $k= 0,1,\ldots,n$ and
  $\min_{k=0}^{n}\left\{  \sum_{i=k}^n a_i u^{i-k}\right\}\ge0$.
\end{proof}

% \textbf{ What's more, some time  the  optimal upper bound satisfying Theorem \ref{thm:log}, is sharper than the bound obtain by Theorem \ref{thm:qup}. }

\begin{theorem}\label{thm:com1}

  Let $p(x)=a_nx^n+a_{n-1}x^{n-1}+\cdots+a_0\ (a_n>0)$ be a polynomial with real coefficients. Let  $u_1$ denote the optimal upper bound of
  positive real roots satisfying Theorem \ref{thm:log} and $u_2$ denote the optimal upper bound of positive real roots satisfying Theorem \ref{thm:qup}, then $u_1\le u_2$ and the  strict inequality can hold.
\end{theorem}

\begin{proof}
  By Theorem \ref{thm:com}, $u_1\le u_2$.
  Let $p(x)=x^2+x-2$. Then $u_2=\sqrt{2}$ and $u_1=1$. So
  $u_1<u_2$ for this $P$.
\end{proof}


\begin{theorem}
  Let $p(x)=a_nx^n+a_{n-1}x^{n-1}+\cdots+a_0\ (V(p)> 0)$ be a polynomial with real coefficients. Let  $u$ denote the output of Algorithm \ref{alg:up} and $u_1$ denote the optimal upper bound of $p$ satisfying
 Theorem \ref{thm:log}. When $u$ is less than or equal to $1$, $u<2u_1$.

\end{theorem}


\begin{algorithm}
\SetAlgoCaptionSeparator{.}
\caption{\less}
\DontPrintSemicolon
\KwIn{ $p=a_nx^n+a_{n-1}x^{n-1}+\cdots+a_1x+a_0\in\ZZ[x], \exists a_i,a_na_i<0 $. }
\KwOut{ true: the positive root bound of $p$ must be less than $1$;\\\ \ \ \ \ \ \ \ \ \ \ \ \ \    false: cannot determine whether the bound is less than $1$. }

$start=n-1$;\;
$lastNeg=0$;\;
$hSign=\sign(a_n)$;\;

\While{$\sign(a_{lastNeg})+hSign\neq 0$ }{
  $lastNeg=lastNeg+1$;\;

}

\While{$\sign(a_{start})+hSign\neq 0$ }{
  $start=start-1$;\;

}
$cfSum=\abs(a_n)$;\;
$i=n-1$;\;
$j=start$;\;
$last=start$;\;
\While{$i\ge lastNeg-1 \text{ and } j\ge lastNeg-1$}
{
	\If{$\sign(cfSum)<0$}
	{
	  \While{$i>last\text{ and } \sign(a_i)\neq hSign$ }{
		$i=i-1$;
	  }

	 \If{$i= = last$}{
	   \Return {false;}
	 }
	 $cfSum=cfSum+\abs(a_i)$;\;
	 $i=i-1$;\;


	}\Else{
	  \If{$j= = lastNeg-1$ }{
		\Return{ true;}


	  }
	  \While{$j\ge lastNeg \text{ and } \sign(a_j)+hSign\neq 0$ }{
		$j=j-1$;
	  }
	  $cfSum=cfSum-\abs(a_j)$;\;
	  $last=j$;\;


	}


}

\Return {true; }
\label{alg:less}
\end{algorithm}



\begin{algorithm}[H]\label{alg:up}
\SetAlgoCaptionSeparator{.}
\caption{\up}
\DontPrintSemicolon
\KwIn{$p=a_nx^n+a_{n-1}x^{n-1}+\cdots+a_1x+a_0\in\ZZ[x], \exists a_i,a_na_i<0. $ }
\KwOut{ an upper bound of the positive roots of $p$. }
$start=n-1$;
$lastNeg=0$;
$hSign=\sign(a_n)$;
$base=1$;\;

\If{$\neg$\less($p$) }{
  return $2$;
}
\While{$\sign(a_{lastNeg})+hSign\neq 0$ }{
  $lastNeg=lastNeg+1$;\;

}

\While{$\sign(a_{start})+hSign\neq 0$ }{
  $start=start-1$;\;

}

$i=n$;\;
\While{$i= = n$}{

  $i=n-1$;\;
  $j=start$;\;
  $cfSum=\abs(a_n)$;\;

	\While{$i\ge lastNeg-1 \text{ and } j\ge lastNeg-1$}{
		\If{$\sign(cfSum)<0$}
		{
		  \While{$i>j\text{ and } \sign(a_i)\neq hSign$ }{
			$i=i-1$;
		 }
		 \If {$i= = j$}{
		   break;
		 }
		 $cfSum=cfSum+\abs(a_i)2^{(n-i)base}$;\;
		 $i=i-1$;\;


	   }\Else{

		 \If{$j= = lastNeg-1$ }{
			$j=lastNeg-2$;
			break;\;

		 }

		\While{$j\ge lastNeg \text{ and } \sign(a_j)+hSign\neq 0$ }{
			$j=j-1$;
		}

		 $cfSum=cfSum-\abs(a_i)2^{(n-j)base}$;\;

		$j=j-1$;\;



	   }

  }
  \If{$j= = lastNeg-2$}{
	$base=base+1$;
	$i=n$;\;

  }


}
\Return{$\frac{1}{2^{base-1}}$; }

\end{algorithm}

\begin{proof}
  In Algorithm \ref{alg:up}, if  $\frac{1}{2^{base}}\ge u_1,$ then $\min_{j=0}^{n}\left\{ \sum_{i=j}^na_i\left( {\frac{1}{2^{base}} }
  \right)^{i-j}\right \}\ge 0$ by the proof of Theorem \ref{thm:log} and thus %, by Line $29$ of Algorithm \ref{alg:up}, $base$ increases $1$ ($ base=base+1$).
  the loop does not terminate at this step.
  So when Algorithm \ref{alg:up} returns, $base$ must satisfy $\frac{1}{2^{base}}<u_1$. Therefore, the output $u=\frac{1}{2^{base-1}}$ and $u<2u_1$.
  Obviously, this algorithm will terminate.

  Furthermore,  $\min_{j=0}^{n-1}\left\{ \sum_{i=j}^na_i\left( {\frac{1}{2^{base-1}} } \right)^{i-j}\right \}\ge 0$ by Theorem \ref{thm:log}. So, $u=\frac{1}{2^{base-1}}$ is an upper bound of $p$.
\end{proof}


\begin{corollary}
  Let $p(x)=a_nx^n+a_{n-1}x^{n-1}+\cdots+a_0$ $ (V(p)> 0)$ be a polynomial with real coefficients. Set $u$ to be the optimal upper bound of positive roots of $P$ satisfying Theorem
  \ref{thm:log}. Then  Algorithm \ref{alg:up} costs at most $O(n\log(u+1))$  additions and multiplications.
\end{corollary}


\section{Tricks}
{\bf Variable substitution}
If $p(x)\in \ZZ[x]$ and $p(x)=p_1(x^k)\ (k>1),$ then substitute $y=x^k$ in $p$. Obviously, $\deg(p_1,y)=\frac{\deg(p,x)}{k}$. We first isolate the real roots of $p_1$ then
obtain the real roots of $p$. We can see in Figure \ref{fig:2} that degree is a key fact affecting the running time. Using this trick, we can greatly reduce the running time of ${\it ChebyshevT}$
and {\it ChebyshevU} when each term of the polynomials is of even degree. The running time on such polynomials can be found in Table \ref{tab:mm2}. The same trick was also taken into account in \cite{johnson06}.

{\bf Incomplete termination check}
If $p(x)\in \ZZ[x]$ and $V(p)= 2$, we may try to check whether the sign of $p(1)$ is the same as the sign of the leading coefficient of $p$. If they are not the same, then $p$  has one positive root in $(0,1)$ and the other one in $(1,+\infty)$. So, we can terminate this subtree. Since the whole \froot\ procedure is a tree and \froot\ spends more than 90 percent of the
total time on computing $T(p)$, this trick may improve the efficiency of the algorithm greatly.