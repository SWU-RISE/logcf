% !Mode\dots ``TeX:UTF-8''

\section{A new upper bound of positive real roots for polynomials}
\label{sec:thm}

One key ingredient of continued fractions based methods is the computation of  the positive real roots' upper bounds. We give in {\em Theorem \ref{thm:log}} a new characteristic of such upper bounds of univariate polynomials. A new algorithm based on this theorem, {\em Algorithm \ref{alg:up}}, is proposed for computing upper bounds of positive real roots.


\begin{theorem}
	\label{thm:log}
  Suppose   $p=a_nx^n+a_{n-1}x^{n-1}+\cdots+a_1x+a_0\ (a_n>0)$  is a univariate polynomial in $x$ with real coefficients.  Then  nonnegative number $u$ is an upper bound of positive roots of $p$ if $u$   satisfies that  $a_i u^{i-j}\ge 0$ for $j=0,\cdots, n $.
\end{theorem}
\begin{proof}
  If $n=0$, then $p$ is a nonzero constant and any positive number is its upper bound of positive roots.

  Otherwise, if $b>u$,  we {\bf claim} that $\sum_{i=j}^na_ib^{i-j}> \sum_{i=j}^na_iu^{i-j}$ for any $j= 0,\ldots,n-1$.

  When $j=n-1$, $\sum_{i=n-1}^na_ib^{i-n+1}-\sum_{i=n-1}^na_iu^{i-n+1}=a_n(b-u)>0.$ Thus, the claim holds.

  Assume the {\bf claim} holds  when $j=k$. When $j=k-1$,  $\sum_{i=k-1}^na_ib^{i-k+1}=\left(\sum_{i=k}^na_ib^{i-k}\right)b+a_{k-1} $. By assumption,
  $\sum_{i=k}^na_ib^{i-k}>\sum_{i=k}^na_iu^{i-k}\ge0$. Since $b>u\ge0$, $\left(\sum_{i=k}^na_ib^{i-k}\right)b>\left (\sum_{i=k}^na_iu^{i-k} \right)u  $
  and $\sum_{i=k-1}^na_ib^{i-k+1}> \sum_{i=k-1}^na_iu^{i-k+1}$. So  $\sum_{i=j}^na_ib^{i-j}> \sum_{i=j}^na_iu^{i-j}$ for any $j= 0,\ldots,n-1$.


  By the above claim,   $p(b)=\sum_{i=0}^na_ib^i>0$ when  $b>u$. Because $b$ is arbitrarily chosen, $u$ is an upper bound of the positive roots of $p$.

\end{proof}

For {\em Example \ref{ex:exp1}},
	\begin{itemize}
			\item  $\sum_{i=6}^6 a_i x^{i-j}$ is $1$.
		\item  $\sum_{i=5}^6 a_i x^{i-j}$ is $x+2$ and $x$ is positive for $x\ge0$.
		\item  	$\sum_{i=4}^6 a_i x^{i-j}$ is $x^2+2x-4$ and its
		two real roots are $-1-\sqrt{5}$ and $-1+\sqrt{5}$.
		\item $\sum_{i=3}^6 a_i x^{i-j}$ is $x^3+2x^2-4x+1$ and its largest real root is  $1$.
		\item  $\sum_{i=2}^6 a_i x^{i-j}$ is $x^4+2x^3-4x^2+x+10$  and  its largest real root is less than $1$.
			\item  $\sum_{i=1}^6 a_i x^{i-j}$ is $x^5+2x^4-4x^3+x^2+10x-5$  and  its largest real root is less than $1$.
				\item  $\sum_{i=0}^6 a_i x^{i-j}$ is $x^6+2x^5-4x^4+x^3+10x^2-5x+5$  and  its largest real root is less than $1$.
	\end{itemize}
	 Hence, $  \sum_{i=j}^6 a_i u^{i-j} \ge 0, j=0\cdots,6$ when   $u=-1+\sqrt{5}$. Employing {\em Theorem \ref{thm:log}}, $-1+\sqrt{5}$ is one  upper bound of positive roots of $p_1$.



The following theorem was given by Akritas et al. in  \cite{akr08}, which computes positive root upper bounds of univariate polynomials.

\begin{theorem}[Akritas-Strzebo\'{n}ski-Vigklas, \cite{akr08}]
	 \label{thm:qup}

  Let $p(x)=a_nx^n+a_{n-1}x^{n-1}+\cdots+a_0\ (a_n>0)$ be a polynomial with real coefficients and let $d(p)$ and $t(p)$ denote the degree and the number of its terms, respectively.

  Moreover, assume that $p(x)$ can be written as\begin{equation}
	p(x)=q_1(x)-q_2(x)+q_3(x)-q_4(x)+\cdots +q_{2m-1}(x)-q_{2m}(x)+g(x)
	\label{eq:1}
  \end{equation}
  where all the coefficients of polynomials $q_i(x)$ $(i=1,2,\ldots,2m)$ and $g(x)$ are positive. In addition,
  assume that for $i=1,2,\ldots,m$ we have
  \begin{equation*}
	q_{2i-1}(x)=c_{2i-1,1}x^{e_{2i-1,1}}+\cdots+c_{2i-1,t_{2i-1}}x^{e_{2i-1,t_{2i-1}}}
  \end{equation*}
  and
  \begin{equation*}
	q_{2i}(x)=b_{2i,1}x^{e_{2i,1}}+\cdots+b_{2i,t_{2i}}x^{e_{2i,t_{2i}}}
  \end{equation*}
  where $e_{2i-1,1}=d(q_{2i-1})$, $e_{2i,1}=d(q_{2i})$, $t_{2i-1}=t(q_{2i-1}),$ and $t_{2i}=t(q_{2i})$   and the exponent of each term in $q_{2i-1}(x)$ is greater than the exponent of each term in $q_{2i}(x)$. If for all indices $i=1,2,\ldots,m$, we have
  \begin{equation*}
	t(q_{2i-1})\ge t(q_{2i}),
  \end{equation*}
  then an upper bound of the values of the positive roots of $p(x)$ is given by

  \begin{equation}\label{eq:2}
	up= \max_{i=1,2,\ldots,m}\left\{ \max_{j=1,2,\ldots,t_{2i}}\left\{ \left( \frac{b_{2i,j}}{c_{2i-1,j}} \right)^{\frac{1}{e_{2i-1,j}- e_{2i,j} } } \right\}  \right\}
  \end{equation}
  for any permutation of the positive coefficients $c_{2i-1,j}, $ $j=1,2,\ldots,t_{2i-1}$.
  Otherwise, for each of the indices $i$ for which we have
  \begin{equation*}
	t_{2i-1}<t_{2i},
  \end{equation*}
  we {\bf break up} one of the coefficients of $q_{2i-1}(x)$ into $t_{2i}-t_{2i-1} +1$ parts, so that now $t(q_{2i} ) = t(q_{2i-1})$ and apply the same formula (\ref{eq:2}) given above.
\end{theorem}

For {\em Example \ref{ex:exp1}} we have
  \begin{eqnarray*}
 q_1&=&x^6+2x^5\\  	
 -q_2&=& -4x^4\\
 q_3&=& x^3+10x^2\\
 -q_4&=& -5x\\
 q_5&=&5.
  	\end{eqnarray*}
A direct application of {\em Theorem \ref{thm:qup}} pairs the terms $\{x^6, -4x^4\}$ of $q_1(x)$ and $q_2(x)$, and ignores the last term
  	of $q_1(x)$. For $q_3(x)$ and $q_4(x)$,   application of {\em Theorem \ref{thm:qup}} pairs the terms $\{x^3, -5x\}$ of them, and igonores the last  term of $q_3(x)$. The resulting upper bound is $\sqrt{5}$. As $\sqrt{5}> -1+\sqrt{5}$, for {\em Example \ref{ex:exp1}}, the upper bound given by {\em Theorem \ref{thm:qup}} is greater than  the upper bound given by {\em Theorem \ref{thm:log}}. For the upper bound, the value is better if the estimated upper bound is smaller. So in this case,  {\em Theorem \ref{thm:log}}  is better than   {\em Theorem \ref{thm:qup}} . In general, we shall show in {\em Theorem \ref{thm:com}} that the optimal bound given by {\em Theorem \ref{thm:log}} is better than that given by {\em Theorem \ref{thm:qup}}.

\begin{theorem}\label{thm:com}
	Let $p(x)=a_nx^n+a_{n-1}x^{n-1}+\cdots+a_0\ (a_n>0)$ be a polynomial with real coefficients. If $u$ is an upper bound of positive roots of $p$ obtained by {\em Theorem \ref{thm:qup}}, then $  \sum_{i=k}^n a_i u^{i-k}\ge0$ for $k=0,\cdots, n$.
	
\end{theorem}
\begin{proof}

	For every $ a_i<0$, by {\em Theorem \ref{thm:qup}}, there exist  $c_{i_1}x^{e_{i_1}}$ and $b_{i_2}x^{e_{i_2}},$  respectively, such that
	$e_{i_1}>e_{i_2}$, $b_{i_2}x^{e_{i_2} }$ is the term $-a_ix^i$
	and $c_{i_1}x^{e_{i_1} }$ is either a whole or a part (broken up by {\em Theorem \ref{thm:qup}}) of a positive term of $p$. Since $u$ satisfies equation (\ref{eq:2}),
	 $c_{i_1}u^{e_{i_1}}\ge b_{i_2}u^{e_{i_2}}$.
So $p(x)$ can be written as $p(x)=\sum (c_{i_1}x^{e_{i_1}}-b_{i_2}x^{e_{i_2}})+g(x)$ where $c_{i_1}> 0,b_{i_2}>0,e_{i_1}> e_{i_2}, e_{{i+1}_1}\ge e_{i_1} $  and all the coefficients of polynomial $g(x)$ are positive.
	 So for  every  $a_j>0$, the sum of coefficient $c_{i_1}$ where  $e_{i_1}=j$ and  the terms of $c_{i_1}x^{e_{i_1}}$    has a corresponding $ b_{i_2}x^{e_{i_2}}$ is less or equal than $a_j$. In other words,
	for every  $a_j>0$,   $\left( \sum_{a_i<0,e_{i_1}=j }c_{i_1} \right)\le a_{j}$. So, if $u\ge0,$
	\[\sum_{i=k}^na_iu^i\ge \sum_{i=k,a_i<0, a_ix^i=-b_{i_2}x^{e_{i_2}}  }^n \left( c_{i_1}u^{e_{i_1}}-
		b_{i_2}u^{e_{i_2}} \right) \] for any $k= 0,1,\ldots,n$.  Since  $c_{i_1}u^{e_{i_1}}\ge b_{i_2}u^{e_{i_2}}$, $\sum_{i=k}^na_iu^i\ge 0$ for
	 $k= 0,1,\ldots,n$. Since $u\ge0$,  $\sum_{i=k}^n a_i u^{i-k}\ge0 $ for any  $k= 0,1,\ldots,n$ and
	$\min_{k=0}^{n}\left\{  \sum_{i=k}^n a_i u^{i-k}\right\}\ge0$.
\end{proof}


%\begin{theorem}\label{thm:com1}
%	
%	Let $p(x)=a_nx^n+a_{n-1}x^{n-1}+\cdots+a_0\ (a_n>0)$ be a polynomial with real coefficients. Let  $u_1$ denote the optimal upper bound of
%	positive real roots satisfying {\em Theorem \ref{thm:log}} and $u_2$ denote the optimal upper bound of positive real roots satisfying {\em Theorem \ref{thm:qup}}, then $u_1\le u_2$ and the  strict inequality can hold.
%\end{theorem}
%
%\begin{proof}
%	By {\em Theorem \ref{thm:com}}, $u_1\le u_2$.
%	Let $p(x)=x^2+x-2$. Then $u_2=\sqrt{2}$ and $u_1=1$. So
%	$u_1<u_2$ for this \rev{$p$}.
%\end{proof}

